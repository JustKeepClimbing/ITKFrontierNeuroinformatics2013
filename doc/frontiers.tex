%%%%%%%%%%%%%%%%%%%%%%%%%%%%%%%%%%%%%%%%%%%%%%%%%%%%%%%%%%%%%%%%%%%%%%%%%%%%%%%%%%%%%%%%%%%%%%%%%%%%%%%%%%%%%%%%%%%%%%%%%%%%%%%%%%%%%%%%
% This is just a template to use when submitting manuscripts to Frontiers, it is not mandatory to use frontiers.cls nor frontiers.tex  %
%%%%%%%%%%%%%%%%%%%%%%%%%%%%%%%%%%%%%%%%%%%%%%%%%%%%%%%%%%%%%%%%%%%%%%%%%%%%%%%%%%%%%%%%%%%%%%%%%%%%%%%%%%%%%%%%%%%%%%%%%%%%%%%%%%%%%%%%

\documentclass{frontiersENG} % for Engineering articles
%\documentclass{frontiersSCNS} % for Science articles
%\documentclass{frontiersMED} % for Medicine articles

\usepackage{url,lineno}
\linenumbers

\usepackage{graphicx}

% Leave a blank line between paragraphs in stead of using \\

\copyrightyear{}
\pubyear{}

\def\journal{Neuroinformatics}%%% write here for which journal %%%
\def\DOI{}
\def\articleType{Research Article}
\def\keyFont{\fontsize{8}{11}\helveticabold }
\def\firstAuthorLast{Sample {et~al.}} %use et al only if is more than 1 author
\def\Authors{Matthew McCormick\,$^{1,*}$,
  Xiaxiao Liu\,$^{1}$,
  Julien Jomier\,$^{2}$,
  Charles Marion\,$^{2}$,
  Joshua Carp\,$^{3}$,
  and Luis Ibanez\,$^1$}
% Affiliations should be keyed to the author's name with superscript numbers and be listed as follows: Laboratory, Institute, Department, Organization, City, State abbreviation (USA, Canada, Australia), and Country (without detailed address information such as city zip codes or street names).
% If one of the authors has a change of address, list the new address below the correspondence details using a superscript symbol and use the same symbol to indicate the author in the author list.
\def\Address{$^{1}$Medical Computing Group, Kitware Inc, Clifton Park, NY, USA\\
$^{2}$Medical Computing Group, Kitware Inc, Lyon, France\\
$^{3}$University of Michigan, Ann Arbor, MI, USA\\
}
% The Corresponding Author should be marked with an asterisk
% Provide the exact contact address (this time including street name and city zip code) and email of the corresponding author
\def\corrAuthor{Luis Ibanez}
\def\corrAddress{Medical Computing Group, Kitware Inc, Clifton Park, NY, USA}
\def\corrEmail{luis.ibanez@kitware.com}

% \color{FrontiersColor} Is the color used in the Journal name, in the title, and the names of the sections


\begin{document}
\onecolumn
\firstpage{1}

\title[ITK Reproducible Research]{ITK Enabling Reproducible Research and Open Science}
\author[\firstAuthorLast ]{\Authors}
\address{}
\correspondance{}
\extraAuth{}% If there are more than 1 additional author, comment this line and uncomment the next one
%\extraAuth{corresponding Author2 \\ Laboratory X2, Institute X2, Department X2, Organization X2, Street X2, City X2 , State XX2 (only USA, Canada and Australia), Zip Code2, X2 Country X2, email2@uni2.edu}
\topic{Research Topic}

\maketitle
\begin{abstract}

\section{}
%As a primary goal, the abstract should render the general significance and conceptual advance of the work clearly accessible to a broad readership. References should not be cited in the abstract.
%See the Summary Table at \\ \url{http://www.frontiersin.org/}\texttt{\journal}\url{/authorguidelines} \\for abstract requirement and length according to article type.
The essential feature of the scientific method is the practice of verification of reproducibility.The large majority of research activity today is focused on generating novelty, and only in exceptional cases, concerned with the verification of reproducibility. The practice of peer-review has been assumed to be a suitable replacement for the verification of reproducibility, a mistake by which experimental work has been replaced by thought experiments and opinion-based evaluations that do little to further the scientific enterprise. This drift has denigrated, what used to be scientific work, back into the practice of the “natural philosophy” in which we simply imagine models of the natural word and evaluate them based on aesthetic appeals and desire for self-consistency.

The Insight Toolkit (ITK) has made it possible to restore the true practice of the scientific method in the field of medical image analysis. By providing a common platform in which image analysis algorithms and processing techniques can be implemented and can be freely disseminated. ITK empowers all to verify the experimental work of image analysis research activities. This of course, requires that researchers adhere to the true practice of the scientific method and publish the full details of their methodology, including the source code, data, parameters and auxiliary documents that are required for a third party to independently repeat the work and verify the findings.

The ITK community created in 2005 a scientific journal, the Insight Journal, truly fulfilling the practice of the scientific method, where all articles are required to provide the full set of source code, data and parameters needed to reproduce the finding of the authors. These materials are immediately made available to readers and reviewers, empowering them to indeed perform such verification with minimal effort and minimal loss of information.

Other Journals, in particular Frontiers, PLoS, and more recently Nature, have embraced this restoration of the true practice of the scientific method.  With the support of the Reproducible Research movement, these progressive publication venues are creating the conditions for a new age of enlightenment in which the methodologies of practical research work will not be subject to secrecy, nor be subject to the veil of suspicion that many incidents of scientific fraud and data manufacturing that have left us with in recent months.

The open source nature of ITK, the open access nature of the Insight Journal, and the public sharing of open data that is enabled by the Midas Platform in the ITK community, are the three pillars of Open Science that are transforming the way scientific research is done today.

The technological challenges of reproducibility have all been solved. We now require a cultural change by which we must make simply unacceptable that any article in the domain of medical image analysis be published without the full set of source code, data and parameters that will enable an independent group to replicate the process and verify or refute the findings.


\tiny
 \keyFont{ \section{Keywords:} Reproducibilty, ITK,Insight Journal, Code Review, Open Science} %All article types: you may provide up to 8 keywords; at least 5 are mandatory.
\end{abstract}


\section{Introduction}

% For Original Research Articles, Clinical Trial Articles, and Technology Reports the introduction should be succinct, with no subheadings.
%
%For Clinical Case Studies the Introduction should include symptoms at presentation, physical exams and lab results.
%
The National Library of Medicine’s Insight Segmentation and Registration Toolkit (ITK) began in 1999 to support analysis of The Visible Human Project. Not only was the project successful in its original objective, but the open source project has seen fantastic success, extending far beyond its original goals as a foundational component of many National Institutes of Health (NIH) research projects and the technology underlying many commercial products in the United States research projects that uses ITK.

The ITK repository currently contains over 2.5 million lines of source code (including third-party code added to the repository). Contributions to this repository can be measured by the number of logical changes made to the code, also known as commits, and the number of source code line additions or deletions.

Ohloh.net is a public directory of open source projects that performs analytics on the code history of communities surrounding projects. According to its Project Cost Calculator, the effort in the toolkit is an estimated 730 person-years, amounting to an estimated cost of 40 million dollars given an average salary of \$55,000 per year.

 As of March 6, 2013, the insight-users mailing list has 2313 subscribers; insight-developers has 510 subscribers. The users’ list averages 282.15 messages per month. The developer list averages 93.92 messages per month (from Feb 2012 - Feb 2013) (LUIS to do:update the number!)



%\begin{methods}
\section{Material \& Methods}

\subsection{source code version control :Git}

\subsection{Open Review System: Gerrit Review}
Recently, recognition of software’s importance in the progression of science has evoked movements such as the Science Code Manifesto \cite{Barnes2010}. While a number of journals are beginning to adopt practices similar to The Insight Journal, where code is submitted along with the article, evaluation tools and procedures for peer review of the code and article are not in parity.  Commonplace practices have evolved to facilitate article review such as evaluation rubrics, instrumentation of the text with line and page numbers for reference during discussion, and a process to distribute an article to reviewers and communicate author replies.  While this provides the reviewers a mechanism to evaluate the reproducibility (i.e., merit) of the article, the technical nature of code solicits greater technical capabilities of the tools and methods used to evaluate its reproducibility.

While this problem is multi-faceted, some progress has been made through ITK’s adoption of the Gerrit Code Review system.  Gerrit is an open source project maintained by the Google Android mobile phone project, and we have contributed fixes and features back to the Gerrit project.

The Gerrit server provides a mechanism to effectively evaluate code changes, obtain and test those changes locally, notification and transmission of the changes and comments for authors and reviewers, and management of the system to accept merges.  Gerrit is a technology that is built around the Git distributed version control system.  With distributed version control system, contributors can independently develop and test patches branched from the ITK master branch. Next, the patch can be shared with the community by pushing the topic branch to the Gerrit server \cite{ITKGerrit}.  Once on the server, the change is publically accessible via a web browser differences are easily identified with a color-highlighted, side-by-side, file-based diff.  Contributors can add reviewers through a web-interface that will auto-complete names any community member that has registered an account with the server.  The reviewers for a given change can be added or removed throughout the review process by community members, and they will be notified via email any new comments or change revisions.  In the Gerrit system, each change is identified by a Change Id, which allows multiple revisions, i.e. Patch Sets, to be uploaded in response to comments.  A discussion of the change between author and reviewers occurs via three mechanisms: overall comments on the change, inline comments, and numerical ratings.  Overall comments conveying general remarks can be added per Patch Set, and the history of comments is retained and easily navigated. Questions and suggestions can be directed at specific codes with the inline comments.  Whether the code can reproducibly be built and pass tests is indicated with a numerical Verified score, and overall evaluation is indicated with a numerical Code Review score.  With the Gerrit Code Review system, reproducibility is improved through continual refinement of the corpus of Insight Toolkit knowledge, i.e. the code repository, through experimentation, i.e. unit tests, and peer review, i.e. code reviews.

\subsection{Open Dashboard}
ITK has a stringent system of quality control that uses a combination of unit testing, regression testing, multi-platform verification, and continuous integration. The testing infrastructure of ITK includes more than 2,400 unit tests, which provide a code coverage of 84\% of ITK code lines. This is well above the industry averages of 50\% code coverage.

The collection of unit tests are executed nightly by computers contributed by community members all around the world, and reporting to a central online dashboard that summarizes the results. This web-based dashboard system (CDash) \cite{ITK dashboard} ensures ITK’s software quality as developers around the world continuously make changes to the code base. Build status and regression test status are visualized in a tabular form. The dashboard is an important coordination and communication tool that empowers developers to share the results of a local test with other developers by pushing them to the online summary pages.

Continuous builds triggered by patches submitted to the Gerrit code review system also give feedback to developers on the impact of recent changes. Nightly builds of the project spanning a wide variety of platforms and configurations ensure that ITK can be built on a diversity of operating systems and hardware.


\subsection{Modularization}
Since its inception in 2000, ITK was designed as a collection of about seven core libraries and about ten third party libraries. This monolithic organization of the code over time led to very large sub-libraries as more classes were added to the toolkit. Once the core code of ITK surpassed the half-a-million lines of code, it became evident that a more modular approach was needed in order to support the future continued growth of the toolkit.

\textbf{Figure 1.}{An illustration on ITK modules' dependencies.}\label{fig:01}

As a result of the modularization, the initial monolithic code base of about 12,000 files was partitioned into more than 100 modules.  And as of Oct of 2013, ITK contains 141 modules in total, four of which are remote modules \cite{remote modules}.  Dependencies \ref{fig:01} between modules were identified and explicitly declared in the CMake-based configuration system. By making the CMake-based configuration system aware of the new ITK modularization, the modularization empowers ITK adopters to select the pieces of ITK that they wanted to use in their own projects at configuration time. Support for integrating externally distributed modules was built into the modularization infrastructure.The Remote Module infrastructure enables fast dissemination of research code through ITK without increasing the size of the main repository.

\subsection{Open Journal: The Insight Journal}

ITK has been conceived as a usable encyclopedia of image analysis algorithms that are of particular utility to the medical imaging community. Given the rapid pace at which technology develops in this area, and the proliferation of both generic and specialized image analysis algorithms, it is important for ITK and its community to continuously update the content of the toolkit by adding new algorithms while simultaneously improving and extending existing ones. To do this, the ITK community relies on contributions made by its members, as they use the toolkit to support their own projects and run into situations where additions and improvement are required for them to achieve their goals. In order to absorb these contributions, the ITK community has used the Insight Journal \cite{Insight Journal} }since 2005. The online Insight Journal publishes practical articles written by developers for developers, and requires those articles to be fully reproducible.

The Insight Journal is a free, open-access publication covering the domain of medical image processing and visualization.It enables community members to publish their contributions to medical projects for open peer-review, full reproducibility verification, and open-rating by readers worldwide.

The journal is very active, and since its inception in 2005, it has published 542 articles with 826 public reviews and has more than 2,500 subscribed readers. The top four articles have been downloaded more than 5,000 times and viewed more than 10,000 times. The Insight Journal is currently the only technical publication in the domain of image analysis that not only allows but also requires verification of reproducibility as part of the submission and review process. The importance of restoring the practice of reproducibility verification in scientific research has resurged in recent years in light of worrisome findings of inconsistency, lack of quality and even fraud in what were otherwise considered to be high-quality publications. By adhering to a reproducibility verification requirement, the Insight Journal ensures that community members get rapid access to reliable publications that include open source software that they can readily use in their projects.

Insight Journal now accepts ITK module submissions. This feature empowers community members to take advantage of the new modular structure in ITK and makes future code integration easier. Readers can then download Insight Journal articles and directly plug-them-in as modules of their local ITK installation.


\subsection{Outreach}
In particular, we have successfully hosted a series of webinars to promote ITK and the new features available in ITKv4. The webinar videos received more than 1,000 plays in total. We have begun moving to hosting more advanced topics, such as “Raspberry Pi Likes ITK” and “Raspberry Pi Likes ITK with VTK”. To empower new developers and lower the barrier of entry for new users, we have initiated a web framework to better distribute ITK examples and put together a large collection of examples on various ITK classes and filters.

In a very focused effort to grow the ITK community, an online space called ITKBarCamp was created to train new community members on the software technologies that are essential to ITK. This space provides a combination of source code, documentation and video tutorials that guide newcomers at their own pace through training materials aimed at honing their software development skills.

One of the very active areas in the ITK BarCamp is a series of participatory, short YouTube.com videos with associated documentation covering various topics related to ITK, including:
\begin{enumeruate}
\item Mastery of the command line
\item Basic C++ programming skills
\item Good software practices, including unit testing
\item Recommended tools and workflows for ITK development.
\end{enumerate}

So far we have created 29 short tutorial videos, which have received 2,234 views and attracted 30 subscribers. ITK Bar Camp materials are hosted in Github as written documentation and video archives. Information on previous and current webinars and hangouts can be found at http://www.itk.org/ITK/resources/webinars.html.

We believe these efforts are crucial to attracting, training, and retaining new community members in the long run, and through that mechanism replenish the community with active members. We plan to continue collecting bar camp tutorials and encourage other community members to contribute to the repository.



\section{Results}

Contribution Statistics
Software Quality assurance

\subsection{Open Review System: Gerrit Review}
In the three years that the community has applied the Gerrit Code review
system, << d['src/gerrit_results.json'].from_json()['changes'] >> changes have
been submitted to the review server and
<< d['src/gerrit_results.json'].from_json()['reviews'] >> were performed.
As a matter of policy, all merged changes should have at least one review,
but the number of iterations on a change varies flexibly depending on the
requirements. This results in a roughly negative exponential distribution in
revisions, as evident in the histogram of revisions in
Figure~\ref{fig:gerrit_patch_set_histogram}.  The highest number of reviews
for a single change was
<< d['src/gerrit_results.json'].from_json()['max_reviews'] >>.

Two direct but notable conclusions follow from this data. First, at least one
other person examined and reproduced a proposed code.  This certainly exceeds
publication systems where the code is never dissemminated, and it likely
exceeds validation systems where the code is published, but there is no
incentive or check that reviewers looked at it or applied it.


Insight Journal
* Number of Submissions
** Number of Revisions
* Number of Reviews
* Integration into the main repository


\section{Discussion}
Software Quality assurance
The code review system helps catch bugs, results in design improvements, assists in the training of new developers, and provides a great communication platform for collaborative development. Since it has so many software quality advantages, code reviews are as critical to ITK as the creation of new patches.


Insight Journal
Past barriers from getting from the Insight Journal into the Toolkit.  Opinions on the
non-blinded review effectiveness.



\section*{Acknowledgement}
Thank NLM, ITK community


\paragraph{Funding\textcolon} The US National Institutes of Health National Library of Medicine (NIH-NLM) has been a primary funder of the Toolkit since its inception 14 years and currently supports the work of authors Ibanez, Liu, and McCormick.




%----- all the figures ----------------
\begin{figure}
  \centering
    \includegraphics[width=0.5\textwidth]{itk_module_dependency.png}
    \caption{An illustration on ITK modules' dependencies.}
    \label{fig:itk_module_dependency}
\end{figure}

\begin{figure}
  \centering
    \includegraphics[width=0.5\textwidth]{gerrit_patch_set_histogram.eps}
    \caption{Histogram of the number of revisions (Patch Sets) for a given change.}
    \label{fig:gerrit_patch_set_histogram}
\end{figure}

\begin{figure}
  \centering
    \includegraphics[width=0.5\textwidth]{gerrit_fix_ups.eps}
    \caption{Fix-up commit percentage before and after peer code review.}
    \label{fig:gerrit_fix_ups}
\end{figure}

\begin{figure}
  \centering
    \showthe\textwidth
    \includegraphics[width=1.0\textwidth]{GerritGraphRender.png}
    \caption{Peer code reviews.  Nodes are individual community members and
      size of the circle at the node is related to logarithm of the number of changes
      created by that community member.  The widths of the edges in this directed
      graph are proportional to the number of reviews performed.}
    \label{fig:gerrit_fix_ups}
\end{figure}




\bibliographystyle{frontiersinSCNS&ENG} % for Science and Engineering articles
%\bibliographystyle{frontiersinMED} % for Medicine articles
\bibliography{test}

\end{document}
